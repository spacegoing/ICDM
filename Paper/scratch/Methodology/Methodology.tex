%% 
%% Template chap2.tex
%% 

\chapter{Methodology}
\label{cha:methodology}

\section{Learning the MRFs}
\label{sec:learning}

\subsection{Structural SVMs}
\label{sec:ssvm}

Traditional Support Vector Machines (SVMs) is used to solve
binary classification problem. \citename{crammer2002algorithmic}
extended SVM into multiclass classifier by generalizing the
concept of \emph{margin} to measure multiclass distances and a
quadratic objective function was constructed. To approach
structural prediction \citename{tsochantaridis2005large} extends
their approach by specifying discriminant functions that exploit
the structure and dependencies within label space $\Y$. In this
section we briefly summarize their work.

Suppose we are given a training set of input-output structure
pairs $S=\{(x_1,y_1),...,(x_n,y_n)\}\in(\X\times \Y)^n$ . We want
to learn a linear prediction rule of the form
\begin{align}
  \label{eq:ssvm}
  f_w(x)=\argmax_{ y\in \Y } [\bw\cdot\phi(x,y)]
\end{align}


where $\phi$ is a joint feature vector that describes the
relationship between input $x$ and structured output $y$, with
$\bw$ being the parameter vector. The optimization problem of
computing this argmax is typically referred to as the \emph{
  inference } problem.

When training Structural SVMs, the parameter vector $\bw$ is
determined by minimizing the (regularized) risk on the training
set $(x_1,y_1),...,(x_n,y_n)$. Risk is measured through a
user-supplied loss function $\Delta(y,\hat{ y })$ that quantifies
how much the prediction $\hat{y}$ differs from the correct output
$y$. Note that $\Delta$ is typically nonconvex and discontinuous
and there are usually exponentially many possible structures
$\hat{y}$ in the output space $\Y$. The Structural SVM
formulation\cite{tsochantaridis2005large} overcomes these
difficulties by replacing the loss function $\Delta$ with a
piecewise linear convex upper bound (margin rescaling)

\begin{align}
  \label{eq:ssvm_bound}
  \Delta(y_i,\hat{ y_i }(\bw))\leq max_{\hat{y}\in\Y}\bigg[\Delta(y_i,\hat{ y })+\bw\cdot\phi(x_i,\hat{y})\bigg]-\bw\cdot\phi(x_i,y_i)
\end{align}
\noindent where $\hat{y}_i(\bw)=\argmax_{y\in\Y}\bw\cdot\phi(x_i,y)$.

To train Structural SVMs we then solve the following convex
optimization problem

\begin{align}
  \label{eq:ssvm_objective}
  min_{\bw}\frac{1}{2}||\bw||^2+\C\sum_{i=1}^{n}\bigg[max_{\hat{y}\in\Y}\big[\Delta(y_i,\hat{ y })+\bw\cdot\phi(x_i,\hat{y})\big]-\bw\cdot\phi(x_i,y_i)\bigg]
\end{align}

Despite the typically exponential size of $\Y$, this optimization
problem can be solved efficiently using cutting-plane or
stochastic gradient methods. Structural SVMs give excellent
performance on many structured prediction tasks, especially when
the model $\phi$ is high-dimensional and it is necessary to
optimize to non-standard loss functions $\Delta$.

\subsection{Learning under the large margin framework}
\label{sec:large_margin}

We now show how the max-margin framework can be used to learn
parameters of our MRFs. We begin by reviewing a variant of the
max-margin framework introduced by
\citename{Tsochantaridis:ICML04} and \citename{Taskar:ICML05}.

Given an energy function $\energy{\by; \btheta} = \btheta^T \!
\phi(\by)$ parameterized as a linear combination of features
$\phi(\by) \in \reals^m$ and weights $\btheta \in \reals^m$, and
a set of $T$ training examples $\{\by_t\}_{t=1}^T$ the max-margin
framework is a principled approach to learning the weights of the
model.

In our formulation we will allow additional linear constraints to be
imposed on the weights of the form $\bG \btheta \geq \bh$, where $\bG
\in \reals^{d \times m}$ and $\bh \in \reals^d$. This is not typically
necessary for max-margin learning, but, as we will see below, is
required for enforcing concavity when learning lower linear envelope
potentials.

Now, let $\Y_t = \{0, 1\}^n$ be the set of all possible assignments
for the $t$-th training example. The (margin-rescaling) max-margin
approach formulates learning as a quadratic programming optimization
problem, $\mmqp{\Y_t}{\bG}{\bh}$:
%
\begin{align}
  & \textrm{minimize} \quad \textstyle \frac{1}{2} \|\btheta\|^2 + \frac{C}{T} \sum_{t=1}^{T} \xi_t
  \label{eqn:maxmarginqp} \\
  & \textrm{subject to} \notag \\
  & \begin{array}{lll}
    & \btheta^T \delta\phi_t(\by) + \xi_t \geq \Delta(\by, \by_t), & \forall t, \by \!\in\! \Y_t, \\
    & \xi_t \geq 0, & \forall t, \\
    & \bG \btheta \geq \bh
  \end{array} \notag
\end{align}
%
where $\delta\phi_t(\by) \triangleq \left(\phi_t(\by) -
\phi_t(\by_t)\right)$ is the difference between feature
representations for some assignment $\by$ and the $t$-th ground-truth
assignment $\by_t$, $C > 0$ is a regularization constant, and
$\Delta(\by, \by_t)$ measures the loss between a ground-truth
assignment $\by_t$ and any other assignment. In our work we use the
Hamming loss, which measures the proportion of variables whose
corresponding assignments disagree. More formally, the Hamming loss is
defined as $\Delta(\by, \by') = \frac{1}{n} \sum_{i=1}^{n} \ind{y_i
  \neq y'_i}$, where $\ind{P}$ is the indicator function taking value
one when $P$ is true and zero otherwise.

\begin{algorithm}[tb]
  \begin{algorithmic}[1]
    \STATE{ {\bf input} training set $\{\by_t\}_{t=1}^{T}$, regularization constant $C > 0$,
      and tolerance $\epsilon \geq 0$}
    \STATE{ {\bf initialize} active constraints set $\A_t = \{ \}$ for all $t$}
    \REPEAT

    \STATE{solve $\mmqp{\A_t}{\bD^2}{\zeros}$ to get $\hat{\btheta}$ and
      $\hat{\bxi}$}

    \STATE{convert from $\hat{\btheta}$ to $(a_k, b_k)$ representation}
    \FOR{each training example, $t = 1, \ldots, T$}
    \STATE{compute $\by_t^\star = \argmin_{\by} E(\by; \hat{\btheta}) - \Delta(\by, \by_t)$}
    \IF{$\hat{\xi}_t + \epsilon \!<\! \Delta(\by_t^\star, \by_t) -
      E(\by_t^\star; \hat{\btheta}) + E(\by_t; \hat{\btheta})$}
    \STATE{$\A_t \leftarrow \A_t \cup \{\by_t^\star\}$}
    \ENDIF
    \ENDFOR
    \UNTIL{no more violated constraints}
    \STATE{ {\bf return} parameters $\hat{\btheta}$}
  \end{algorithmic}
  \caption{\label{alg:learning} Learning lower linear envelope MRFs.}
\end{algorithm}

The number of constraints in the QP is exponential in the number of
variables, and a standard approach to solving the max-margin QP is by
adding constraints incrementally. Briefly, at each iteration the
algorithm checks for the most violated constraint (for each training
example), using \emph{loss-augmented inference}, and, if found, adds
it to the constraint set. The algorithm terminates when no more
violated constraints are found (see \algref{alg:learning}).

Note that while we use the Hamming loss in this work, the loss
function $\Delta(\by, \by_t)$ in \eqnref{eqn:maxmarginqp} can be more
general. For example, \citet{Pletscher:AISTATS12} recently showed that
certain higher-order losses can be reduced to binary pairwise
supermodular functions. In this way the loss function factors over the
same terms as in the energy function with the addition of auxiliary
variables. Since the loss function is subtracted from the energy
function during loss-augmented inference, the supermodular loss
becomes a submodular objective and therefore admits tractable
minimization.

Our test for the most violated constraints (lines 7 and 8) can be
performed exactly ($\Delta(\by, \by_t)$ decomposes as a sum of
unary terms). If the test succeeds, then $\by_t^\star$ cannot
already be in $\A_t$. It is now added (line 9). Since there are
only finitely many constraints, this happens at most $2^n - 1$
times (per training example), and the algorithm must eventually
terminate. On termination there are no more violated constraints,
hence the parameters are optimal.

Unfortunately, as our proof suggests, it may take exponential
time for the algorithm to reach convergence with $\epsilon = 0$.
\citename{Tsochantaridis:JMLR05} showed, however, that for
$\epsilon > 0$ and no additional linear constraints (\ie $\bG =
\zeros$, $\bh = \zeros$) max-margin learning within a dual
optimization framework will terminate in a polynomial number of
iterations. Their result can be extended to the case of
additional linear constraints.



% % \clearpage
% % \cleardoublepage


% % %%% Local Variables:
% % %%% mode: latex
% % %%% TeX-master: "../thesis"
% % %%% End:
