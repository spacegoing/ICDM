\message{ !name(../thesis.tex)}%%
%% Template thesis.tex
%%
\documentclass[twoside,doublespace,onecolumn,11pt,a4paper]{book}
\usepackage[palatino]{StyFiles/anuthesis}
\usepackage{graphicx}
\usepackage{StyFiles/thesis}
\usepackage{makeidx}
% \usepackage{StyFiles/doublespace}

% Chuck Added
\usepackage[toc,page]{appendix}
\usepackage{StyFiles/fancyhdr}
% Gould Configurations

% For figures
% \ifCLASSINFOpdf
% \usepackage[pdftex]{graphicx}
% \DeclareGraphicsExtensions{.jpg,.png}
% \else
% \usepackage[dvips]{graphicx}
% \DeclareGraphicsExtensions{.eps}
% \fi

% For citations
\usepackage[sort,numbers]{StyFiles/natbib}
\renewcommand{\citename}{\citet}
\renewcommand{\cite}{\citep}
\usepackage{StyFiles/natbibspacing}

% For maths
\usepackage[cmex10]{amsmath}
\usepackage{amssymb,amsthm}

% For algorithms
\usepackage{StyFiles/algorithm}
\usepackage{StyFiles/algorithmic}

% For Hyperlinks
\usepackage{hyperref}
% fix problem between hyperref and algorithmic
\newcommand{\theHalgorithm}{\arabic{algorithm}}

% For captions
\usepackage[font=small,labelfont=bf]{caption}
\usepackage[font=footnotesize]{subfig}

% My macros
\usepackage{StyFiles/sg-macros}

\newtheorem{thm}{Theorem}[section]
\newtheorem{cor}[thm]{Corollary}
\newtheorem{lem}[thm]{Lemma}
\newtheorem{prop}[thm]{Proposition}
\newtheorem{obs}[thm]{Observation}
\newtheorem{defn}[thm]{Definition}

\newcommand{\mmqp}[3]{\textrm{\sc MaxMarginQP}\!\left(\{\by_t, #1\}_{t=1}^{T}, #2, #3\right)}

% correct bad hyphenation here
\usepackage{StyFiles/hyphenat}
\hyphenation{op-tical net-works semi-conduc-tor}


%%%%%%%%%%%%%%%%%%%%%%%%%%%%%%%%%%%%%%%%%%%%%%%%%%%%%%%%%%%%%%%%%%%%%%%
%% Preamble
\title{Latent Structural SVM Learning for Lower Linear
  Envelope Potentials in Binary Markov Random Fields}
\author{Chang Li} \date{\today}

\renewcommand{\thepage}{\roman{page}}

\makeindex
\begin{document}

\message{ !name(RelatedWorks/RelatedWorks.tex) !offset(-76) }
%% 
%% 
%% 

\chapter{Related Work and Background}
\label{cha:RelatedWorks}

\section{Related Work}
\subsection{Markov Random Fields}
\label{sec:MRF}
\emph{Markov Random Fields} are also known as \emph{undirected
  graphical model} can be seen as a regularized joint
log-probability distribution of arbitrary non-negative functions
over a set of maximal cliques of the
graph~\cite{bishop:2006:PRML}. Let $C$ denotes a maximal clique
in one graph and $\by_C$ denotes the set of variables in that
clique. Then the joint distribution can be written as:
\begin{align}
  p(\by)=\frac{1}{Z}\prod_{C}{\Psi_C(\by_C)}
\end{align}
\noindent where $\Psi$ is called \emph{potential functions} which
can be defined as any non-negative functions and
$Z=\sum_{\by}\prod_{C}{\Psi_C(\by_C)}$ which is a normalization
constant. To infer labels which best explains input data set, we
can find the \emph{maximum a posteriori} (MAP) labels by solving
$\by^*=\argmax_{\by}p(\by)$. Because potential functions are
restricted to be non-negative, it gives us more flexible
representations by taking exponential of those terms. Thus the
joint distribution becomes:
\begin{align}
  p(\by)=\frac{1}{Z}exp(-\sum_{C}{E_C(\by_C)})
\end{align}
\noindent where $E$ is called \emph{energy functions} which can be
arbitrary functions. Therefore, \emph{maximum a posteriori}
problem is equivalent to \emph{energy minimization} problem,
which is also known as \emph{inference}:
\begin{align}
  \by^*=\argmax_{\by}p(\by)=\argmin_{\by}(-\sum_{C}{E_C(\by_C)})
\end{align}
To optimize the performance we can also consider a weighted
version of energy functions. In order to do this we can decompose
energy functions over nodes $\N$, edges $\E$ and higher order
cliques $\C$~\cite{Szummer:ECCV08} then add weights on them
accordingly. Let $\bw$ be the vector of parameters and $\phi$ be
arbitrary feature function, then the energy can be decomposed as
a set linear combinations of weights and feature vectors:

\begin{align}
  \label{eq:energyfunction_UPH}
  E(\by;\bw)=\sum_{i\in \N}{\bw_i^U\phi^U(\by_i)}+
  \sum_{(i,j)\in \E}{\bw_{ij}^P\phi^P(\by_i,\by_j)}+
  \sum_{\by_C\in \C}{\bw_C^H\phi^H(\by_C)}
\end{align}

\noindent where $U$ denotes \emph{unary} terms, $P$ denotes
\emph{pairwise} terms and $H$ denotes \emph{higher order} terms
(when $|C|>2$ namely each clique contains more than two
variables).

A weight vector $\bw$ is more preferable if it gives the
ground-truth assignments $\by_t$ less than or equal to energy
value than any other assignments $y$:

\begin{align}
E(y_t,w)\leq E(y,w)~ \text{,~}\forall y \neq y_t
\text{,~} y\in \Y
\end{align}


Thus the goal of \emph{learning} MRFs is to learn the parameter
vector $\bw^*$ which returns the lowest energy value for the
ground-truth labels $y_t$ relative to any other assignments
$y$~\cite{Szummer:ECCV08}:

\begin{align}
\bw^* = argmax_{\bw}(E(y_t,w)-E(y,w))~ \text{,~}\forall y \neq y_t
\text{,~} y\in \Y
\end{align}

We have introduced three main research topics of MRFs:
definition of \emph{energy function} (potential functions),
\emph{inference} problem (MAP or energy minimization) and
\emph{learning} problem. As for energy function, our work focus
on a class of higher-order potentials defined as a concave
piecewise linear function which is known as lower linear envelope
potentials over a clique of binary variables. It has been raising
much interest due to its capability of encoding consistency
constraints over large subsets of pixels in an
image~\cite{Kohli:CVPR07,Nowozin:2011}.

\citename{kohli2009robust} proposed a method to represent a class
of higher order potentials with lower (upper) linear envelope
potentials. By introducing auxiliary
variables~\cite{Kohli:CVPR10}, they reduced the linear
representation to a pairwise form and proposed an approximate
algorithm with standard linear programming methods. However, they
only show an exact inference algorithm on at most three terms.
Following their routine, \citename{gouldlearning} extended their
method to a weighted lower linear envelope with arbitrary many
terms which can be solved with an efficient algorithm. They
showed the energy function with auxiliary variables is submodular
by transforming it into a quadratic pseudo-Boolean
form~\cite{Boros:MATH02} and how
\emph{graph-cuts}~\cite{Hammer:1965, Boykov:ICCV01, Freedman:CVPR05} like
algorithm can be applied to do exact \emph{inference}.

\citename{gouldlearning} solved \emph{learning} problem of lower
linear envelope under the max margin
framework~\cite{tsochantaridis2005large}. In their work they
pointed out the potential relationship between their auxiliary
representation and latent SVM~\cite{yu2009learning}. Our work is
closely based on their research. We continue to use the higher
order energy function and inference algorithm developed in their
previous work~\cite{Gould:ICML2011} and extend their max margin
learning algorithm to include latent variables. The learning
algorithm we use is an extension of max margin framework which is
known as ``latent structural SVM''~\cite{yu2009learning}.

% \subsection{Latent Structural SVMs}
% \label{sec:latent-struct-svms}

% The max-margin
% framework~\cite{Taskar:ICML05,tsochantaridis2005large} is a
% principled approach to learn the weights of pairwise MRFs.
% \citename{Szummer:ECCV08} adapted this framework to optimize
% parameters of pairwise MRFs inferred by graph-cuts method. In our
% previous work \citename{gouldlearning} extended this framework
% with additional linear constraints which enforces concavity on
% weights thus can be used for learning lower linear envelope
% potentials.

% In this section we introduce \emph{latent structural
%   SVM}~\cite{yu2009learning} which extends the max-margin
% framework by encoding latent information in feature vector. In
% section~\ref{sec:learning} we will show how this framework can be
% adapted to learn parameters for higher order energy function with
% latent variables.

% Given an a linear combination of features vector $\phi(\bx ,\by)
% \in \reals^m$ and weights $\btheta \in \reals^m$, and a set of
% $n$ training examples $\{\by_i\}_{i=1}^n$ max-margin framework
% can be used to solve optimized solution $\btheta^*$. To include
% unobserved information in the model, Yu\cite{yu2009learning}
% extended the joint feature function\cite{tsochantaridis2005large}
% $\phi(\mathbf{x},\mathbf{y}) $ with a latent variable
% $\mathbf{h}\in \mathcal{H}$ to
% $\phi(\mathbf{x},\mathbf{y},\mathbf{h}) $. So the inference
% problem becomes
% \begin{align}
%   \label{eq:latent_ssvm_linearcomb}
%   f_\theta(x) = \argmax_{(\mathbf{y} \times \mathbf{h}) \in \mathcal{Y}
%   \times \mathcal{H}} \theta\cdot\phi(\mathbf{x},\mathbf{y},\mathbf{h})
% \end{align}

% Accordingly, the loss function can be extended as

% $$
% \Delta((\mathbf{y}_i,\mathbf{h}^*_i(\theta)),(\mathbf{\hat{y}}_i(\theta),\mathbf{\hat{h}}_i(\theta)))
% $$

% \noindent where

% \begin{align}
%   \label{eq:latentssvm_full_inf}
%  (\mathbf{\hat{y}}_i(\theta),\mathbf{\hat{h}}_i(\theta))=\argmax_{(\mathbf{y}
%   \times \mathbf{h}) \in \mathcal{Y} \times \mathcal{H}}
% \theta\cdot\phi(\mathbf{x}_i,\mathbf{y_i},\mathbf{h})
% \end{align}

% \begin{align}
%   \label{eq:latentssvm_latent_inf}
%   \mathbf{h}^*_i(\theta) = \argmax_{\mathbf{h} \in \mathcal{H}} \theta \cdot
%   \phi(\mathbf{x}_i,\mathbf{y}_i,\mathbf{h})
% \end{align}

% The loss function under this formulation measures difference
% between the inferred result pair $(\mathbf{\hat{y}}_i(\theta),
% \mathbf{\hat{h}}_i(\theta))$ and the pair $(\mathbf{y}_i(\theta),
% \mathbf{h}_i^*(\theta))$ which best explains the training data.
% However, under this formulation the ``loss augmented inference''
% used in structural SVMs\cite{tsochantaridis2005large} to remove
% the complexity cannot be performed due to the dependence of loss
% function $\Delta$ on hidden variables $\mathbf{h}^*_i(\theta)$.
% \citename{yu2009learning} argued that in real world applications
% hidden variables are usually intermediate results and are not
% required as an output\cite{yu2009learning}. Therefore, the loss
% function can only focus on the inferenced hidden variables
% $\mathbf{\hat{h}}_i(\theta)$ which leads to:

% $$
% \Delta((\mathbf{y}_i,\mathbf{h}^*_i(\theta)),(\mathbf{\hat{y}}_i(\theta),\mathbf{\hat{h}}_i(\theta)))
% =
% \Delta(\mathbf{y}_i,\mathbf{\hat{y}}_i(\theta),\mathbf{\hat{h}}_i(\theta))
% $$

% Thus the upper bound used in standard structural
% SVMs\cite{tsochantaridis2005large} can be extended to:

% \begin{align}
%   \Delta((\mathbf{y}_i,\mathbf{h}^*_i(\theta)),(\mathbf{\hat{y}}_i(\theta),\mathbf{\hat{h}}_i(\theta)))
%   &\leq \bigg(\max_{(\mathbf{\hat{y}} \times \mathbf{\hat{h}}) \in
%     \mathcal{Y} \times \mathcal{H}}
%     [\theta\cdot\Psi(\mathbf{x}_i,\mathbf{\hat{y}},\mathbf{\hat{h}}) +
%     \Delta(\mathbf{y}_i,\mathbf{\hat{y}},\mathbf{\hat{h}})]\bigg)\\
%   &-\max_{\mathbf{h} \in \mathcal{H}} \theta \cdot
%     \Psi(\mathbf{x}_i,\mathbf{y}_i,\mathbf{h})
% \end{align}

% Hence the optimization problem for Structural SVMs with latent
% variables becomes

% \begin{align}
% \label{eq:latent_ssvm_object}
%   \min_\theta\bigg(\frac{1}{2}\|\theta\|^2+
%   C\sum_{i=1}^{n}\big(\max_{(\mathbf{\hat{y}} \times
%   \mathbf{\hat{h}}) \in \mathcal{Y} \times \mathcal{H}}
%   [\theta\cdot\Psi(\mathbf{x}_i,\mathbf{\hat{y}},\mathbf{\hat{h}}) +
%   \Delta(\mathbf{y}_i,\mathbf{\hat{y}},\mathbf{\hat{h}})]\big)\bigg)\\
%   -C\sum_{i=1}^{n}\big(\max_{\mathbf{h} \in \mathcal{H}} \theta \cdot
%   \Psi(\mathbf{x}_i,\mathbf{y}_i,\mathbf{h})\big)\nonumber
% \end{align}

% \noindent which is a difference of two convex functions. Problem
% of this formulation can be solved using the Concave-Convex
% Procedure (CCCP)\cite{yuille2002concave} which is guaranteed to
% converge to a local minimum. \citename{yu2009learning} proposed a
% two stages algorithm. In the first step the latent variable
% $\bh_i^*$ which best explains training pair $(\bx_i, \by_i)$ is
% found by solving equation~\eqref{eq:latentssvm_latent_inf}. This
% step is also called the ``latent variable completion'' problem.
% In the second step $\bh_i^*$ is used as completely observed to
% substitute $\bh$ in equation~\eqref{eq:latent_ssvm_object}.
% Therefore, solving equation~\eqref{eq:latent_ssvm_object} is
% equivalent to solve the standard structural SVM problem.

% Auxiliary variables was introduced to help representing lower
% linear envelope potentials in an energy-minimization
% setting~\cite{Kohli:CVPR10}. In order to adapt the energy
% function to max margin framework, \citename{Gould:ICML2011}
% approximated the energy function using equally spaced
% break-points thus removed those auxiliary variables. In this
% thesis we try to optimize the energy function exactly by
% introducing auxiliary variables back into the feature vector and
% solving the learning problem using the latent structural SVM
% framework. We will present this in detail in
% section~\ref{sec:learning}.

% \input RelatedWorks/Background.tex


% \clearpage
% \cleardoublepage


% %%% Local Variables:
% %%% mode: latex
% %%% TeX-master: "../thesis"
% %%% End:


\message{ !name(../thesis.tex) !offset(-184) }

\end{document}

%%% Local Variables: 
%%% mode: latex
%%% TeX-master: "thesis"
%%% End: 



